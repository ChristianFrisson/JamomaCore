% Comments at top of file (fold) 

%% Alexander: Using the template from NIME 2007, since the new one is not available yet.

% This is "sig-alternate.tex" V1.3 OCTOBER 2002
% This file should be compiled with V1.6 of "sig-alternate.cls" OCTOBER 2002
%
% This example file demonstrates the use of the 'sig-alternate.cls'
% V1.6 LaTeX2e document class file. It is for those submitting
% articles to ACM Conference Proceedings WHO DO NOT WISH TO
% STRICTLY ADHERE TO THE SIGS (PUBS-BOARD-ENDORSED) STYLE.
% The 'sig-alternate.cls' file will produce a similar-looking,
% albeit, 'tighter' paper resulting in, invariably, fewer pages.
%
% ----------------------------------------------------------------------------------------------------------------
% This .tex file (and associated .cls V1.6) produces:
%       1) The Permission Statement
%       2) The Conference (location) Info information
%       3) The Copyright Line with ACM data
%       4) NO page numbers
%
% as against the acm_proc_article-sp.cls file which
% DOES NOT produce 1) thru' 3) above.
%
% Using 'sig-alternate.cls' you have control, however, from within
% the source .tex file, over both the CopyrightYear
% (defaulted to 2002) and the ACM Copyright Data
% (defaulted to X-XXXXX-XX-X/XX/XX).
% e.g.
%\CopyrightYear{2005}   %will cause 2002 to appear in the copyright line.
%\crdata{XXX}   %will cause XXX to appear in the copyright line.
%
% ---------------------------------------------------------------------------------------------------------------
% This .tex source is an example which *does* use
% the .bib file (from which the .bbl file % is produced).
% REMEMBER HOWEVER: After having produced the .bbl file,
% and prior to final submission, you *NEED* to 'insert'
% your .bbl file into your source .tex file so as to provide
% ONE 'self-contained' source file.
%
% ================= IF YOU HAVE QUESTIONS =======================
% Questions regarding the SIGS styles, SIGS policies and
% procedures, Conferences etc. should be sent to
% Adrienne Griscti (griscti@acm.org)
%
% Technical questions _only_ to
% Gerald Murray (murray@acm.org)
% ===============================================================
%
% For tracking purposes - this is V1.3 - OCTOBER 2002

% Comments at top of file (end) 

\documentclass{sig-alternate}
\usepackage{listings}      
\usepackage{hyperref}
\usepackage{url}

\begin{document}
\lstset{language=XML, basicstyle=\small, stringstyle=\ttfamily}

% Title part (fold) 
%
% --- Author Metadata here ---
\CopyrightYear{2008}   %will cause 2005 to appear in the copyright line.
\crdata{Copyright remains with the author(s).}
\conferenceinfo{NIME08,}{ Genova, Italy}
%\CopyrightYear{2001} % Allows default copyright year (2000) to be over-ridden - IF NEED BE.
%\crdata{0-12345-67-8/90/01}  % Allows default copyright data (0-89791-88-6/97/05) to be over-ridden - IF NEED BE.
% --- End of Author Metadata ---

\title{Differentiating Values and Properties in an Open Sound Control Namespace}
\subtitle{}
%
% You need the command \numberofauthors to handle the "boxing"
% and alignment of the authors under the title, and to add
% a section for authors number 4 through n.
%
% Up to the first three authors are aligned under the title;
% use the \alignauthor commands below to handle those names
% and affiliations. Add names, affiliations, addresses for
% additional authors as the argument to \additionalauthors;
% these will be set for you without further effort on your
% part as the last section in the body of your article BEFORE
% References or any Appendices.

\numberofauthors{1}
%
% You can go ahead and credit authors number 4+ here;
% their names will appear in a section called
% "Additional Authors" just before the Appendices
% (if there are any) or Bibliography (if there
% aren't)

% Put no more than the first THREE authors in the \author command
\author{
%
% The command \alignauthor (no curly braces needed) should
% precede each author name, affiliation/snail-mail address and
% e-mail address. Additionally, tag each line of
% affiliation/address with \affaddr, and tag the
%% e-mail address with \email.
\alignauthor Authors... \\
       \affaddr{institutions...}\\
}

%%%%%%%%%%%%%%%%
\maketitle


% Title part (end) 


% Hyphenations (fold)

\hyphenation{ja-mo-ma}
% Hyphenations (end)

\begin{abstract}

The paper suggests an approach for creating structured Open Sound Control (OSC) messages by separating the addressing of node values and node properties.  This includes a mechanism for querying values and properties.  As a result, it is possible to address complex nodes inside of more complex tree structures using an OSC namespace.  This is particularly useful for creating flexible communication in modular systems.  A prototype implementation is presented and discussed.

\end{abstract}

\keywords{Jamoma, OSC, Open Sound Control, standardization}



%%%%%%%%%%%%%%%%%%%%%%
%%%%%%%%%%%%%%%%%%%%%%
%%%%%%%%%%%%%%%%%%%%%%

\section{Introduction} % (fold)
\label{sec:introduction}

Open Sound Control \footnote{\url{http://www.opensoundcontrol.org}} is a protocol for transmitting messages that address nodes using a familiar "slash" notation.  It has been adopted by the computer music community as a de facto standard for communication both within a single process and between separate processes.  OSC does not prescribe any particular namespace standardization, only a message formatting standardization.

The authors are involved in developing OSC namespaces for the Jamoma\footnote{\url{http://www.jamoma.org}} project.
Jamoma is a modular system for developing high-level modules in the Max/MSP/Jitter environment \cite{Place:2006}.  Communication within and between Jamoma modules is handled using the Open Sound Control protocol. 

As Jamoma's modular structures grew more complex, the authors found the flat namespace conventions of OSC to be inadequate for addressing our constructs.  Specifically, we found OSC to be ideal for specifying an address to a node.  However, it became increasing unclear what to do once the address reached this said node.  The problem is exacerbated when the node itself implements its own OSC namespace.

% section introduction (end)



%%%%%%%%%%%%%%%%%%%%%%
%%%%%%%%%%%%%%%%%%%%%%
%%%%%%%%%%%%%%%%%%%%%%

\section{Representing complex structures in an OSC namespace} % (fold)
\label{sec:representing_complex_structures_in_an_OSC_namespace}


%%%%%%%%%%%%%%%%%%%%%%
\subsection{Open Sound Control}

Open Sound Control was developed in the late 1990s at CNMAT and has emerged as the de facto standard for communication among computers, sound synthesizers, and other multimedia devices in the computer music research community \cite{Wright:2003}. The original idea of OSC is that it is tree-structured into a hierarchy called the \emph{address space} where each of the nodes has a symbolic name and is a potential destination of OSC messages \cite{Wright:2003}. Oppositely to the well-defined structure of MIDI, the \emph{open} nature of OSC means that the address space is defined and created by the "implementor’s idea of how these features should be organized." For example:

\texttt{/voices/3/freq 440.0}

This open approach has made OSC useful in a number of different situations and adaptable to situations not foreseen by developers \cite{Wright:2005}.  However, this lack of standardization in namespace schemas is also probably a major reason that OSC has not gained a more widespread in commercial software applications.


%%%%%%%%%%%%%%%%%%%%%%
\subsection{OSC Namespace Standardization}

There are a number of projects ongoing which attempt to standardize namespaces for various uses of Open Sound Control. A partial listing of these follow.\\



\noindent \textbf{Occam} \footnote{\url{www.mat.ucsb.edu/$\sim$c.ramakr/illposed/occam.html}} takes OSC messages and converts them to MIDI. It exports a MIDI source to CoreMIDI which can be used in any Mac OS X application that accepts MIDI. It broadcats the existence of this OSC-to-MIDI service using Rendezvous (Zero-Conf).\\    
\textbf{LIBOSCQS}\footnote{\url{http://liboscqs.sourceforge.net}} is a library to provide a Query System and Service Discovery for applications using the Open Sound Control (OSC) protocol\cite{Schmeder:2004oscqs}. This page contains an introduction to LIBOSCQS. \\
\textbf{Integra}, an EU Culture 2000 project\footnote{\url{http://www.integralive.org}}  aiming at developing "a new software environment to make music with live electronics," and to modernize works that use old technology \cite{Bullock:2007}. Part of the development has been concerning creating a client/server model.  The Integra OSC address space suggests using the : (colon) to only set messages without outputting, to avoid feedback. They also include a .get message to query for values.\\ 
\textbf{SuperCollider}, a text based audio synthesis programming language is using OSC for internal communication in its client/server model.\\ 
\textbf{Iannix} \footnote{\url{http://sourceforge.net/projects/iannix}} is a graphical editor of 
multidimensional and multi-formal scores, a kind of poly-temporal meta-sequencer, based on the former UPIC created by Iannis Xenakis." The current implementation is built around a client/server model where all communication is handled using OSC. \cite{Coduys:2004}\\
\textbf{Open Sound Control 2.0 proposal} -- Developers of the Lemur multitouch interface presented a draft for a 2.0 version of Open Sound Control during an OSC developer meeting at 
NIME 2006 \cite{Jazzmutant:2006}, and has later changed the suggestion slightly \cite{Jazzmutant:2007}.\\ 
\textbf{OSCQS} \cite{Habets:2005}\\
\textbf{TUIO}, an OSC address space for handling data from the Reactable \cite{Kaltenbrunner:2005} was been developed at MTG at Pompeu Fabra.\\
\textbf{Copperlan} TODO:WHAT IS THAT?\\
\textbf{McGill Mapper} \cite{Malloch:2007}\\


\subsection{Syntactic Definitions} % (fold)
\label{sub:litterature_review}
   
In actual practice, these various independent efforts at standardizing namespaces incorporate syntactic elements with conflicting meanings as compared to each other.  There are some commonalities to these efforts and the problems that they try to address.  Specifically:
\begin{itemize}
	\item How to get the value of a node (vs. setting the value of a node) 
	\item How to access properties of a node (as opposed to the value of the node)
\end{itemize}

It is clear from a review of these sundry efforts that additional syntax is needed for clarifying function, address, or both when developing a complex OSC namespace.

In Jamoma, there are a wide variety of cases where a value may be set for a given node in the OSC namespace.  However, the node also has additional properties which define the behavior of that node.  For example, a node may represent the temperature such as:

\texttt{/path/to/the/temperature 32}

However, we may wish to set a property of the node so that it knows how to interpret the temperature.  Is it specified in Celsius or Fahrenheit?  Is this a shift from an existing value, or is setting the value directly?  Before we send the value, perhaps we should query the node to find out what an acceptable range is for this node.

This problem of accessing properties of a node, as opposed to the value of a node, is particularly troublesome using the standard OSC addressing conventions (using only slashes to navigate the tree).  One example of a system parallel to OSC for representing and sending data over a network is XML.


\subsection{XML} % (fold)
\label{sub:xml}

Extensible Markup Language (XML)\footnote{\url{http://www.w3.org/XML/}} is a particularly relevant analogue to Open Sound Control.  XML defines a means for formatting data, but not the data or the anything specific to the dataspace itself CITATION OF THE XML SPEC http://www.w3.org/TR/2006/REC-xml-20060816/ .  

A number of standardized namespaces using XML have gained wide adoption, including Scalable Vector Graphics (SVG)\footnote{\url{http://www.w3.org/Graphics/SVG/}}, XHTML\footnote{\url{http://www.w3.org/TR/xhtml1/}} and SOAP\footnote{\url{http://www.w3.org/TR/soap/}}.  SOAP is of particular interest because it is designed as a protocol for exchanging structured information.

Using XML, information is encapsulated into elements.  These elements could be structured such that they are analogous to an OSC message. For the purpose of this discussion we will treat them as such.  Using XML elements, one way to represent the above temperature example is thus:

\begin{lstlisting}
  <path>
    <to>
      <the>
        <temperature> 
          32
        </temperature>
      </the>
    </to>
  </path>
\end{lstlisting}

This is clearly more cumbersome to manually type than the Open Sound Control message, it is more work for the receiving processor to parse, and it uses more bandwidth as well. However, XML elements are able not only to express a value (content in xml parlance) between the tags, but also they can provide properties (attributes) to the node. For example, we may provide the type of temperature we are specifying:

\begin{lstlisting}
  <path>
    <to>
      <the>
        <temperature unit='Kelvin'>
          32
        </temperature>
      </the>
    </to>
  </path>
\end{lstlisting}

We suggest a model where it is possible to fork an OSC address to access not only the value of the node, but also the properties of that node, much like what is possible in other existing models such as XML.  Different than XML, we will propose that they can be addressed independently rather than simultaneously.


\subsection{The colon separator} % (fold)
\label{sub:the_colon_separator}

In addition to the ASCII symbols already reserved for specific purposes within the OSC protocol \cite{Wright:1997}, we introduce the colon "\texttt{:}" as a separator between the OSC address of a node and the namespace for accessing the properties of the node:

\texttt{<parameter address> <value>}

\texttt{<parameter address>:<property address> <value>}

The former message sets the value of the node just as it would using the existing OSC conventions. The latter sets a property of the value.  Again using temperature as an example, we can send two messages: one for setting the unit property, and one for setting the value.

\texttt{/path/to/the/temperature:/unit Fahrenheit}

\texttt{/path/to/the/temperature 212}

This usage of the colon has precedent in POSIX paths when addressing remote filesystems.  For example, scp uses the following format to locate a file on a remote server:

\texttt{user@host.com:/path/to/file}

In our case, we are indeed using the colon to separate OSC namespaces, one of which is the address to a node and one of which is the address within that remote node.

Section~\ref{sec:prototype_implementation} provides an illustration of the ideas suggested here. In the following discussion the address of the value will be omitted for the sake of brevity; e.g.\\ 
\texttt{/computer/module/parameter:property}  
will be abbreviated as \texttt{:property}.



% subsection the_colon_separator (end)

% section representing_complex_structures_in_an_OSC_namespace (end)


%%%%%%%%%%%%%%%%%%%%%%
%%%%%%%%%%%%%%%%%%%%%%
%%%%%%%%%%%%%%%%%%%%%%

\section{Jamoma as a prototype implementation} % (fold)
\label{sec:prototype_implementation}

The general concepts introduced in the previous section has formed the basis for the implementation of a standardized namespace addressing properties of values in the Jamoma framework for Max/MSP \cite{Place:2006}. The following description of the implementation is meant to serve as an illustration of the concept, as well as indicating how it might provide the user with extended and structured control of available values.

Jamoma distinguish between module \emph{parameters} and \emph{messages}. Parameters alter the state of the module. When querying or setting the state of the module, parameter values need to be retrieved or set. In contrast messages are stateless. One example of a message would be a request to open a reference file for the module. Apart from the difference in terms of state, messages and parameters are implemented and behave the same way, and the following discussion is equally valid for both. In the following discussion a \emph{value} can be a module parameter or message.



%%%%%%%%%%%%%%%%%%%%%%

\subsection{Value type} % (fold)
\label{sub:type}

The type of the value can be specified. Possible types are \emph{none}, \emph{boolean}, \emph{integer}, \emph{float}, \emph{symbols} and \emph{list}. If one do not want to restrict the type of the value, it can be set to \emph{generic}. The \emph{none} type is only valid for messages. Some of the properties below will only be valid for certain types of values. The type property is accessed thus:

\texttt{:type}

\texttt{:type/get}

% subsection type (end)



%%%%%%%%%%%%%%%%%%%%%%

\subsection{Controlling the value itself} % (fold)
\label{sub:controlling_the_value_itself}

In addition to setting the value directly, it can be set and retrieved as a property. If the value is of integer, float or list type it can also be stepwise increased or decreased. If so the size of the steps is itself a property:

\texttt{:value}

\texttt{:value/get}

\texttt{:inc}

\texttt{:dec} 

\texttt{:value/stepsize}

\texttt{:value/stepsize/get}

% subsection controlling_the_value_itself (end)



%%%%%%%%%%%%%%%%%%%%%%

\subsection{Controlling the range} % (fold)
\label{sub:range}

For integer, float and list values a range can be specified. This can be useful for setting up autoscaling mappings from one value to another, or for clipping the output range. The clipping property can be \emph{none}, \emph{low}, \emph{high} or \emph{both}. The range properties are accessed thus:

\texttt{:range}

\texttt{:range/get}

\texttt{:range/clipmode}

\texttt{:range/clipmode/get}

% subsection range (end)



%%%%%%%%%%%%%%%%%%%%%%

\subsection{Filtering of repetitions} % (fold)
\label{sub:filtering_of_repetitions}

It is sometimes useful to filter repetitions to avoid redundant processing. The \emph{boolean} repetitions property is accessed thus:

\texttt{:repetitions}

\texttt{:repetitions/get}

% subsection filtering_of_repetitions (end)



%%%%%%%%%%%%%%%%%%%%%%

\subsection{Ramping to new values} % (fold)
\label{sub:ramping_to_new_values}

The ability to smoothly move from one value to another is fundamental to any kind of transition ands transformation of musical or artistic material. Jamoma offers the possibilty of ramping from the current to a new value in a set amount of time. While the OSC message

\texttt{/myComputer/myModule/myParameter 1.0}

will set the parameter value to $1.0$ immediately, the message

\texttt{/myComputer/myModule/myParameter 1.0 ramp 2000}

will cause the value to ramp to $1.0$ over 2000 milliseconds. Ramping in Jamoma works with messages and parameters of type integer, float and list.

Jamoma offers vastly extended possibilities in how ramping can be done as compared to Max. In Jamoma the process of ramping is made up from the combination of two components: A driving mechanism cause calculations of new values at desired intervals during the ramp, while a set of functions offers a set of curves for the ramping. Both components are implemented as C++ APIs, and can easily be extended with new ramp or function \emph{units}, expanding the range of possible ramping modes.



\subsubsection{The Jamoma RampLib API} % (fold)
\label{ssub:the_ramp_lib}

The Jamoma RampLib API provides a means by which to create and use \emph{ramp units} in Jamoma.  A ramp unit is a self-contained algorithm that can slide from an existing value to a new value over a specified amount of time according to different timing mechanisms. Each ramp unit is implemented in the form of two C++ files: a source file and a header file that provides an interface for the source file. Currently four such ramp units are implemented:

\begin{itemize}

	\item \emph{none} - jumps immediately to the new value. Typically used for values where ramping do not make sense.

	\item \emph{scheduler} - use the Max internal clock to generate new values at fixed time intervals.

	\item \emph{queue} - ramping using the Max queue, updating values whenever the processor has free capacity to do so.

	\item \emph{async} - only calculate new values when requested to do so. This might be used in video processing modules to calculate fresh values immediately before processing the next video image or matrix.
	
\end{itemize}

When a new ramp is started, the ramp unit internally use a normalized ramping value, increasing linearly from $0.0$ to $1.0$ over the duration of the ramp. Whenever the ramp unit is to provide a new value, it updates the normalized ramping value, and pass it to a Function Unit as described in Section~\ref{ssub:the_function_lib}. The normalized value returned is then scaled to the range defined by the start and end values for the ramp, and passed on to the module.

% subsubsection the_ramp_lib (end)


\subsubsection{The Jamoma FunctionLib API} % (fold)
\label{ssub:the_function_lib}

The Jamoma FunctionLib API provides normalized mappings of values $x \in [0,1]$ to $y \in [0,1]$ according to functions $y = f(x)$. The FunctionLib can easily be expanded by introducing new functions in the form of two C++ files: a source file and a header file that provides an interface for the source file.

Currently five functions are implemented: 

\begin{itemize}

	\item Linear: $y = x$.

	\item Cosine: $y = - \frac{1}{2} \cdot cos(x \cdot \pi ) + \frac{1}{2} $.

	\item Lowpass series: $y[n] = y[n-1] \cdot k + x[n] \cdot (1-k)$, \\ where $k$ is a feedback coefficient.

	\item Power function: $ y = x^{k} $, where the parameter $k$ can be set.

	\item Hyperbolic tangent: $ y = c \cdot (tanh(a\cdot(x-b)) - d) $, \\ where coefficients $a$, $b$, $c$, $d$ depends on the width and offset of the curve.
	
\end{itemize}

There are plans to introduce exponential functions.

% subsubsection the_function_lib (end)



\subsubsection{OSC namespace for ramping properties} % (fold)
\label{ssub:osc_namespace_for_ramping_properties}

Ramping properties are addressed using \texttt{:ramp/drive} and \texttt{:ramp/function} OSC nameclasses. In addition to the ablity of setting or getting current ramp driving mechanism, it might be useful to have the module return a list of all available ramp units. This can be done by means of a \texttt{/dump} message:

\texttt{:ramp/drive}

\texttt{:ramp/drive/get}

\texttt{:ramp/drive/dump}

Some ramp units might have additional parameters that can be controlled by the user. For instance the user can control how often the \emph{scheduler} ramp unit is to update; the granularity of the ramp.  In general a ramp unit parameter \texttt{/foo} might be accessed thus:


\texttt{:ramp/drive/parameter/foo}

\texttt{:ramp/drive/parameter/foo/get}

If the current ramp unit do not have a \texttt{/foo} parameter, the message will be ignored by the ramp unit. It is possible to query what parameters are available for the current ramp unit with the \texttt{/dump} message:

\texttt{:ramp/drive/parameter/dump}

The function used for the ramp can be set or queried in much the same way. It is also possible to request information on available function units with the \texttt{/dump} message:

\texttt{:ramp/function}

\texttt{:ramp/function/get}

\texttt{:ramp/function/dump}

Some function units might have additional coefficients influencing the shape of the curve, e.g. the exponent of the \emph{power} function unit can be set. In general a function unit parameter \texttt{/bar} might be accessed thus:

\texttt{:ramp/function/parameter/bar}

\texttt{:ramp/function/parameter/bar/get}

If the current function unit do not have a \texttt{/bar} parameter, the message will be ignored by the function unit. It is possible to query what coefficients are available for the current function unit with the \texttt{/dump} message:

\texttt{:ramp/function/parameter/dump}

% subsubsection osc_namespace_for_ramping_properties (end)



%%%%%%%%%%%%%%%%%%%%%%

\subsection{Controlling the user interface} % (fold)
\label{sub:controlling_the_user_interface}

In certain applications the CPU overhead of continuously updating the graphical user interface whenever parameter or message values change might become a burden, competing for CPU with e.g. video processing algorithms. If the user do not need continuous visual feedback on updated values of parameters or messages, the GUI for the parameter or message can be frozen, freeing up the processor and GPU for tasks considered more important:

\texttt{:ui/freeze}

\texttt{:ui/freeze/get}

A parameter or message that has its GUI frozen can be forced to update and refresh the displayed value once by means of the message:

\texttt{:ui/refresh}

% subsection controlling_the_user_interface (end)

% subsection ramping_to_new_values (end)



%%%%%%%%%%%%%%%%%%%%%%

\subsection{Description} % (fold)
\label{sub:description}

The description property is a string providing a text description of the parameter. In Jamoma this is used for auto-generating online documentation of the modules. It can also be used for building modules that retrieve the total namespace of all Jamoma modules used, and provide interactive documentation of available parameters. The descriptions property is accessed as:

\texttt{:description}

\texttt{:description/get}

% subsection description (end)



%%%%%%%%%%%%%%%%%%%%%%

\subsection{Retrieving all properties of a value} % (fold)
\label{sub:retrieving_all_properties_of_a_value}

The discussion so far has illustrated what properties are available for values in Jamoma, and how the can be accessed one by one. Jamoma also offers the possibility of retrieving the total namespace for one value with the message: 

\texttt{:namespace/dump}

In addition current settings for all properties can be retrieved by the message:

\texttt{:properties/dump}

% subsection retrieving_all_properties_of_a_value (end)

% section prototype_implementation (end)



%%%%%%%%%%%%%%%%%%%%%%
%%%%%%%%%%%%%%%%%%%%%%
%%%%%%%%%%%%%%%%%%%%%%

\section{Discussion and further work} % (fold)
\label{sec:discussion_and_further_work}

4 ramp units times 5 function units = 20 ramping modes

ramp units can be used for other scheduled processes as well

Possibility of expanding ramp units as low frequency oscillators

function units can be used elsewhere, e.g. for mapping

Audio rate ramp unit.

DataspaceLib

Querying - we propose a different system to the Lemur OSC2 draft

Ramp Lib and Function Lib can be used outside the context of jcom.parameter and jcom.value: jcom.map and jcom.ramp



% section discussion_and_further_work (end)




%\end{document}  % This is where a 'short' article might terminate



%%%%%%%%%%%%%%%%%%%%%%
%%%%%%%%%%%%%%%%%%%%%%
%%%%%%%%%%%%%%%%%%%%%%

%ACKNOWLEDGMENTS are optional
\section{Acknowledgments} % (fold)
\label{sec:acknowledgments}

All Jamoma developers and users for valuable contributions. 
iMAL

% section acknowledgments (end)



%%%%%%%%%%%%%%%%%%%%%%
%%%%%%%%%%%%%%%%%%%%%%
%%%%%%%%%%%%%%%%%%%%%%

% Bibliography (fold)
%
% The following two commands are all you need in the
% initial runs of your .tex file to
% produce the bibliography for the citations in your paper.
\bibliographystyle{abbrv}
\bibliography{jamoma-nime2008}  % the name of the Bibliography in this case
% You must have a proper ".bib" file
%  and remember to run:
% latex bibtex latex latex
% to resolve all references
%
% Bibliography (end)


\balancecolumns % GM July 2000
% That's all folks!
\end{document}
