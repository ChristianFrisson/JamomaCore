%% Alexander: Using the template from NIME 2007, since the new one is not available yet.

% This is "sig-alternate.tex" V1.3 OCTOBER 2002
% This file should be compiled with V1.6 of "sig-alternate.cls" OCTOBER 2002
%
% This example file demonstrates the use of the 'sig-alternate.cls'
% V1.6 LaTeX2e document class file. It is for those submitting
% articles to ACM Conference Proceedings WHO DO NOT WISH TO
% STRICTLY ADHERE TO THE SIGS (PUBS-BOARD-ENDORSED) STYLE.
% The 'sig-alternate.cls' file will produce a similar-looking,
% albeit, 'tighter' paper resulting in, invariably, fewer pages.
%
% ----------------------------------------------------------------------------------------------------------------
% This .tex file (and associated .cls V1.6) produces:
%       1) The Permission Statement
%       2) The Conference (location) Info information
%       3) The Copyright Line with ACM data
%       4) NO page numbers
%
% as against the acm_proc_article-sp.cls file which
% DOES NOT produce 1) thru' 3) above.
%
% Using 'sig-alternate.cls' you have control, however, from within
% the source .tex file, over both the CopyrightYear
% (defaulted to 2002) and the ACM Copyright Data
% (defaulted to X-XXXXX-XX-X/XX/XX).
% e.g.
%\CopyrightYear{2005}   %will cause 2002 to appear in the copyright line.
%\crdata{XXX}   %will cause XXX to appear in the copyright line.
%
% ---------------------------------------------------------------------------------------------------------------
% This .tex source is an example which *does* use
% the .bib file (from which the .bbl file % is produced).
% REMEMBER HOWEVER: After having produced the .bbl file,
% and prior to final submission, you *NEED* to 'insert'
% your .bbl file into your source .tex file so as to provide
% ONE 'self-contained' source file.
%
% ================= IF YOU HAVE QUESTIONS =======================
% Questions regarding the SIGS styles, SIGS policies and
% procedures, Conferences etc. should be sent to
% Adrienne Griscti (griscti@acm.org)
%
% Technical questions _only_ to
% Gerald Murray (murray@acm.org)
% ===============================================================
%
% For tracking purposes - this is V1.3 - OCTOBER 2002

\documentclass{sig-alternate}

\begin{document}
%
% --- Author Metadata here ---
\CopyrightYear{2008}   %will cause 2005 to appear in the copyright line.
\crdata{Copyright remains with the author(s).}
\conferenceinfo{NIME08,}{ Genova, Italy}
%\CopyrightYear{2001} % Allows default copyright year (2000) to be over-ridden - IF NEED BE.
%\crdata{0-12345-67-8/90/01}  % Allows default copyright data (0-89791-88-6/97/05) to be over-ridden - IF NEED BE.
% --- End of Author Metadata ---

\title{Developing a Structured OSC Namespace for Jamoma}
\subtitle{}
%
% You need the command \numberofauthors to handle the "boxing"
% and alignment of the authors under the title, and to add
% a section for authors number 4 through n.
%
% Up to the first three authors are aligned under the title;
% use the \alignauthor commands below to handle those names
% and affiliations. Add names, affiliations, addresses for
% additional authors as the argument to \additionalauthors;
% these will be set for you without further effort on your
% part as the last section in the body of your article BEFORE
% References or any Appendices.

\numberofauthors{1}
%
% You can go ahead and credit authors number 4+ here;
% their names will appear in a section called
% "Additional Authors" just before the Appendices
% (if there are any) or Bibliography (if there
% aren't)

% Put no more than the first THREE authors in the \author command
\author{
%
% The command \alignauthor (no curly braces needed) should
% precede each author name, affiliation/snail-mail address and
% e-mail address. Additionally, tag each line of
% affiliation/address with \affaddr, and tag the
%% e-mail address with \email.
\alignauthor Authors... \\
       \affaddr{institutions...}\\
}

%%%%%%%%%%%%%%%%
\maketitle
\begin{abstract}

The paper suggests an approach to create structured OSC messages, separating the addressing of computers and modules, from the parameters and attributes of the modules. This includes a system for querying values and parameters for creating flexible communication in modular systems. A prototype implementation is presented and discussed.

\end{abstract}

\keywords{Jamoma, OSC, standardization}


%%%%%%%%%%%%%%%%%%%%%%
%%%%%%%%%%%%%%%%%%%%%%
%%%%%%%%%%%%%%%%%%%%%%
\section{Introduction}

Jamoma\footnote{http://www.jamoma.org} is a system for developing high-level modules in the Max/MSP/Jitter environment, consisting of a recommendation and an implementation of that recommendation  \cite{Place:2006}. Most of the recent development of Jamoma has focused on improving core functionality, including flexible mapping between modules and adding ramping, function and unit conversion possibilities. 

Communication in and between Jamoma modules is being handled through the Open Sound Control (OSC)\footnote{http://www.opensoundcontrol.org} protocol. As the messaging between has grown more complex we have found that the current messaging structure of OSC is not ideal for our usage. The paper will start with an overview of some related research into development of the OSC protocol. This is followed by a suggestion for a structured approach to extending the current suggestions for OSC namespace creation. Finally, a prototype implementation in Jamoma is presented and discussed.


%%%%%%%%%%%%%%%%%%%%%%
%%%%%%%%%%%%%%%%%%%%%%
%%%%%%%%%%%%%%%%%%%%%%
\section{Literature Review}


%%%%%%%%%%%%%%%%%%%%%%
\subsection{Open Sound Control}

Open Sound Control (OSC) was developed in the late 1990s at CNMAT and has emerged as the de facto standard for communication among computers, sound synthesizers, and other multimedia devices in the computer music research community \cite{Wright:2003}. 


The original idea of OSC is that it is tree-structured into a hierarchy called the \emph{address space} where each of the nodes has a symbolic name and is a potential destination of OSC messages \cite{Wright:2003}. Oppositely to the well-defined structure of MIDI, the \emph{open} nature of OSC means that the address space is defined and created by the "implementor’s idea of how these features should be organized." 



“/voices/3/freq” 

This open approach has made OSC useful in a number of different situations, but it is also probably the major reason that OSC has not gained a more widespread usage outside the computer music research community. 

%%%%%%%%%%%%%%%%%%%%%%
\subsection{Occam}


Occam takes OSC messages and converts them to MIDI. It exports a MIDI source to CoreMIDI which can be used in any Mac OS X application that accepts MIDI. It broadcats the existence of this OSC-to-MIDI service using Rendezvous (Zero-Conf).

http://www.mat.ucsb.edu/~c.ramakr/illposed/occam.html


%%%%%%%%%%%%%%%%%%%%%%
\subsection{liboscqs}

LIBOSCQS\footnote{http://liboscqs.sourceforge.net/} is a library to provide a Query System and Service Discovery for applications using the Open Sound Control (OSC) protocol. This page contains an introduction to LIBOSCQS.


%%%%%%%%%%%%%%%%%%%%%%
\subsection{McGill Mapper}

\cite{Malloch:2007}


%%%%%%%%%%%%%%%%%%%%%%
\subsection{OSCQS}

\cite{Habets:2005}

%%%%%%%%%%%%%%%%%%%%%%
\subsection{Integra}

The Integra\footnote{http://www.integralive.org} project is a EU Culture 2000 project aiming at developing "a new software environment to make music with live electronics," and to modernise works that use old technology \cite{Bullock:2007}. Part of the development has been concerning creating a client/server model. 

The Integra OSC address space suggests using the : (colon) to only set messages without outputting, to avoid feedback. 

They also include a .get message to query for values. 


%%%%%%%%%%%%%%%%%%%%%%
\subsection{SuperCollider}

The text based computer music programming language SuperCollider is using OSC for internal communication in its client/server model. 


%%%%%%%%%%%%%%%%%%%%%%
\subsection{Iannix}

"Iannix \footnote{http://sourceforge.net/projects/iannix} is a graphical editor of multidimensional and multi-formal scores, a kind of poly-temporal meta-sequencer, based on the former UPIC created by Iannis Xenakis." The current implementation is built around a client/server model where all communication is handled using OSC. 

\cite{Coduys:2004}


%%%%%%%%%%%%%%%%%%%%%%
\subsection{TUIO}

The MTG at Pompeu Fabra has developed TUIO, an OSC address space for handling data from the Reactable \cite{Kaltenbrunner:2005}.

Copperlan

%%%%%%%%%%%%%%%%%%%%%%
\subsection{Open Sound Control 2.0 suggestion}

Developers of the Lemur multitouch interface presented a draft for a 2.0 version of Open Sound Control during an OSC developer meeting at NIME 2006 \cite{Jazzmutant:2006}, and has later changed the suggestion slightly \cite{Jazzmutant:2007}. 

The topic of query has been brought up and been suggested as follows: 




%%%%%%%%%%%%%%%%%%%%%%
%%%%%%%%%%%%%%%%%%%%%%
%%%%%%%%%%%%%%%%%%%%%%
\section{A Structured Approach}


%%%%%%%%%%%%%%%%%%%%%%
\subsection{Separating addresses and parameters}

We suggest a model where it is possible to separate the recipient of addresses from the parameters to be used by the recipient (i.e. a module in Jamoma). 

This could lead to a general address space like: 

\begin{small}
\begin{verbatim}
/computer/module.N/parameter:/
\end{verbatim}
\end{small}



%%%%%%%%%%%%%%%%%%%%%%
\subsection{Classes}




%%%%%%%%%%%%%%%%%%%%%%
%%%%%%%%%%%%%%%%%%%%%%
%%%%%%%%%%%%%%%%%%%%%%
\section{Prototype Implementation}

Our discussion and testing has happened within the Jamoma framework for Max/MSP

%%%%%%%%%%%%%%%%%%%%%%
\subsection{Separators}


Having discussed different ways of designing a message like this one, we seem to have agreed that the meaning of the different separating signs used need to be precisely defined:

\begin{itemize}

	\item. (dot): Used to indicate instances of a specific class.
	\item / (slash): Used to indicate branching according to the OSC specification.
	\item : (colon): Colon is used to split the total OSC message into two parts. The first part is describing where you want to go. The second part describe what you want to access there.
\end{itemize}


%%%%%%%%%%%%%%%%%%%%%%
\subsection{Namespace structure}


%% This should probably be in a table if included, but just adding it here as a draft

		Message                                               Category       Description

		- [ ] /module/parameter
		    - [ ] General for the module
		        - [ ] :namespace/dump
		        - [ ] :dump/values
		    - [ ] value stuff                                                These ones are not fixed.
		        - [ ] :value/get                              value          get the value of the parameter
		        - [ ] :value                                  value          set the value (why would you want to do it this way?)
		        - [ ] :value/range                            range
		        - [ ] :value/range/get                        range
		        - [ ] :value/range/clipmode                   range
		        - [ ] :value/range/clipmode/get               range
		        - [ ] :value/step/inc                         value          Increase by one step
		        - [ ] :value/step/dec                         value          Decrease by one step
		        - [ ] :value/step/size                        value          Set the step size
		        - [ ] :value/step/size/get                    value          Get the step size
		        - [ ] :value/unit
		        - [ ] :value/type                             type
		        - [ ] :value/type/get                         type
		    - [ ] Conditioning
		        - [ ] :repetitions                            repetitions
		        - [ ] :repetitions/get                        repetitions
		    - [ ] ramp stuff - including drive and function
		        - [ ] :ramp/drive                             ramp/drive     Select what mechanism is driving the ramp
		        - [ ] :ramp/drive/get                         ramp/drive
		        - [ ] :ramp/drive/parameter                   ramp/drive     Pass additional parameters to the driving mechanism
		        - [ ] :ramp/drive/parameter/get               ramp/drive
		        - [ ] :ramp/function                          ramp/drive     Select what mapping function to use for the ramp
		        - [ ] :ramp/function/get                      ramp/function
		        - [ ] :ramp/function/parameter                ramp/function  Pass additional parameters to the function
		        - [ ] :ramp/function/parameter/get            ramp/function
		        - [ ] :ramp/function/parameter/names/dump     ramp/function  Get the name of all parameters for the currently used function
		        - [ ] :ramp/function/parameter width          ramp/function  This is an example (valid for the tanh function)
		        - [ ] :ramp/function/parameter/get width      ramp/function  This is an example (valid for the tanh function)
		    - [ ] ui
		        - [ ] :ui/freeze                              ui
		        - [ ] :ui/freeze/get                          ui

%%%%%%%%%%%%%%%%%%%%%%
%%%%%%%%%%%%%%%%%%%%%%
%%%%%%%%%%%%%%%%%%%%%%
\section{Discussion}


%%%%%%%%%%%%%%%%%%%%%%
\subsection{Future Work}



%\end{document}  % This is where a 'short' article might terminate

%ACKNOWLEDGMENTS are optional
\section{Acknowledgments}
All Jamoma developers and users for valuable contributions. 


%
% The following two commands are all you need in the
% initial runs of your .tex file to
% produce the bibliography for the citations in your paper.
\bibliographystyle{abbrv}
\bibliography{jamoma-nime2008}  % the name of the Bibliography in this case
% You must have a proper ".bib" file
%  and remember to run:
% latex bibtex latex latex
% to resolve all references
%


\balancecolumns % GM July 2000
% That's all folks!
\end{document}
