%% Alexander: Using the template from NIME 2007, since the new one is not available yet.

% This is "sig-alternate.tex" V1.3 OCTOBER 2002
% This file should be compiled with V1.6 of "sig-alternate.cls" OCTOBER 2002
%
% This example file demonstrates the use of the 'sig-alternate.cls'
% V1.6 LaTeX2e document class file. It is for those submitting
% articles to ACM Conference Proceedings WHO DO NOT WISH TO
% STRICTLY ADHERE TO THE SIGS (PUBS-BOARD-ENDORSED) STYLE.
% The 'sig-alternate.cls' file will produce a similar-looking,
% albeit, 'tighter' paper resulting in, invariably, fewer pages.
%
% ----------------------------------------------------------------------------------------------------------------
% This .tex file (and associated .cls V1.6) produces:
%       1) The Permission Statement
%       2) The Conference (location) Info information
%       3) The Copyright Line with ACM data
%       4) NO page numbers
%
% as against the acm_proc_article-sp.cls file which
% DOES NOT produce 1) thru' 3) above.
%
% Using 'sig-alternate.cls' you have control, however, from within
% the source .tex file, over both the CopyrightYear
% (defaulted to 2002) and the ACM Copyright Data
% (defaulted to X-XXXXX-XX-X/XX/XX).
% e.g.
%\CopyrightYear{2005}   %will cause 2002 to appear in the copyright line.
%\crdata{XXX}   %will cause XXX to appear in the copyright line.
%
% ---------------------------------------------------------------------------------------------------------------
% This .tex source is an example which *does* use
% the .bib file (from which the .bbl file % is produced).
% REMEMBER HOWEVER: After having produced the .bbl file,
% and prior to final submission, you *NEED* to 'insert'
% your .bbl file into your source .tex file so as to provide
% ONE 'self-contained' source file.
%
% ================= IF YOU HAVE QUESTIONS =======================
% Questions regarding the SIGS styles, SIGS policies and
% procedures, Conferences etc. should be sent to
% Adrienne Griscti (griscti@acm.org)
%
% Technical questions _only_ to
% Gerald Murray (murray@acm.org)
% ===============================================================
%
% For tracking purposes - this is V1.3 - OCTOBER 2002

\documentclass{sig-alternate}

\begin{document}
%
% --- Author Metadata here ---
\CopyrightYear{2008}   %will cause 2005 to appear in the copyright line.
\crdata{Copyright remains with the author(s).}
\conferenceinfo{NIME08,}{ Genova, Italy}
%\CopyrightYear{2001} % Allows default copyright year (2000) to be over-ridden - IF NEED BE.
%\crdata{0-12345-67-8/90/01}  % Allows default copyright data (0-89791-88-6/97/05) to be over-ridden - IF NEED BE.
% --- End of Author Metadata ---

\title{Developing a Structured OSC Namespace for Jamoma}
\subtitle{}
%
% You need the command \numberofauthors to handle the "boxing"
% and alignment of the authors under the title, and to add
% a section for authors number 4 through n.
%
% Up to the first three authors are aligned under the title;
% use the \alignauthor commands below to handle those names
% and affiliations. Add names, affiliations, addresses for
% additional authors as the argument to \additionalauthors;
% these will be set for you without further effort on your
% part as the last section in the body of your article BEFORE
% References or any Appendices.

\numberofauthors{1}
%
% You can go ahead and credit authors number 4+ here;
% their names will appear in a section called
% "Additional Authors" just before the Appendices
% (if there are any) or Bibliography (if there
% aren't)

% Put no more than the first THREE authors in the \author command
\author{
%
% The command \alignauthor (no curly braces needed) should
% precede each author name, affiliation/snail-mail address and
% e-mail address. Additionally, tag each line of
% affiliation/address with \affaddr, and tag the
%% e-mail address with \email.
\alignauthor Authors... \\
       \affaddr{institutions...}\\
}

%%%%%%%%%%%%%%%%
\maketitle
\begin{abstract}

The paper suggests an approach to create structured OSC messages, separating the addressing of computers and modules, from the parameters and attributes of the modules. This includes a system for querying values and parameters for creating flexible communication in modular systems. A prototype implementation is presented and discussed.

\end{abstract}

\keywords{Jamoma, OSC, standardization}


%%%%%%%%%%%%%%%%%%%%%%
%%%%%%%%%%%%%%%%%%%%%%
%%%%%%%%%%%%%%%%%%%%%%
\section{Introduction}

Jamoma\footnote{http://www.jamoma.org} is a system for developing high-level modules in the Max/MSP/Jitter environment, consisting of a recommendation and an implementation of that recommendation  \cite{Place:2006}. Most of the recent development of Jamoma has focused on improving core functionality, including flexible mapping between modules and adding ramping, function and unit conversion possibilities. 

Communication in and between Jamoma modules is being handled through the Open Sound Control (OSC)\footnote{http://www.opensoundcontrol.org} protocol. As the messaging between has grown more complex we have found that the current messaging structure of OSC is not ideal for our usage. The paper will start with an overview of some related research into development of the OSC protocol. This is followed by a suggestion for a structured approach to extending the current suggestions for OSC namespace creation. Finally, a prototype implementation in Jamoma is presented and discussed.


%%%%%%%%%%%%%%%%%%%%%%
%%%%%%%%%%%%%%%%%%%%%%
%%%%%%%%%%%%%%%%%%%%%%
\section{Related Work}


%%%%%%%%%%%%%%%%%%%%%%
\subsection{Open Sound Control}


%%%%%%%%%%%%%%%%%%%%%%
\subsection{Integra}

Integra\footnote{http://www.integralive.org} \cite{Bullock:2007} 



%%%%%%%%%%%%%%%%%%%%%%
\subsection{Open Sound Control 2.0 suggestion}

Developers of the Lemur multitouch interface presented a draft for a 2.0 version of Open Sound Control during an OSC developer meeting at NIME 2006 \cite{Jazzmutant:2006}, and has later changed the suggestion slightly \cite{Jazzmutant:2007}.


%%%%%%%%%%%%%%%%%%%%%%
\subsection{UPF}



%%%%%%%%%%%%%%%%%%%%%%
%%%%%%%%%%%%%%%%%%%%%%
%%%%%%%%%%%%%%%%%%%%%%
\section{A Structured Approach}



%%%%%%%%%%%%%%%%%%%%%%
%%%%%%%%%%%%%%%%%%%%%%
%%%%%%%%%%%%%%%%%%%%%%
\section{Prototype Implementation}



%%%%%%%%%%%%%%%%%%%%%%
%%%%%%%%%%%%%%%%%%%%%%
%%%%%%%%%%%%%%%%%%%%%%
\section{Discussion}


%%%%%%%%%%%%%%%%%%%%%%
\subsection{Future Work}


%\end{document}  % This is where a 'short' article might terminate

%ACKNOWLEDGMENTS are optional
\section{Acknowledgments}
All Jamoma developers and users for valuable contributions. 


%
% The following two commands are all you need in the
% initial runs of your .tex file to
% produce the bibliography for the citations in your paper.
\bibliographystyle{abbrv}
\bibliography{jamoma-nime2008}  % the name of the Bibliography in this case
% You must have a proper ".bib" file
%  and remember to run:
% latex bibtex latex latex
% to resolve all references
%


\balancecolumns % GM July 2000
% That's all folks!
\end{document}
