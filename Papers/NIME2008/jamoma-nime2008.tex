% Comments at top of file (fold) 

%% Alexander: Using the template from NIME 2007, since the new one is not available yet.

% This is "sig-alternate.tex" V1.3 OCTOBER 2002
% This file should be compiled with V1.6 of "sig-alternate.cls" OCTOBER 2002
%
% This example file demonstrates the use of the 'sig-alternate.cls'
% V1.6 LaTeX2e document class file. It is for those submitting
% articles to ACM Conference Proceedings WHO DO NOT WISH TO
% STRICTLY ADHERE TO THE SIGS (PUBS-BOARD-ENDORSED) STYLE.
% The 'sig-alternate.cls' file will produce a similar-looking,
% albeit, 'tighter' paper resulting in, invariably, fewer pages.
%
% ----------------------------------------------------------------------------------------------------------------
% This .tex file (and associated .cls V1.6) produces:
%       1) The Permission Statement
%       2) The Conference (location) Info information
%       3) The Copyright Line with ACM data
%       4) NO page numbers
%
% as against the acm_proc_article-sp.cls file which
% DOES NOT produce 1) thru' 3) above.
%
% Using 'sig-alternate.cls' you have control, however, from within
% the source .tex file, over both the CopyrightYear
% (defaulted to 2002) and the ACM Copyright Data
% (defaulted to X-XXXXX-XX-X/XX/XX).
% e.g.
%\CopyrightYear{2005}   %will cause 2002 to appear in the copyright line.
%\crdata{XXX}   %will cause XXX to appear in the copyright line.
%
% ---------------------------------------------------------------------------------------------------------------
% This .tex source is an example which *does* use
% the .bib file (from which the .bbl file % is produced).
% REMEMBER HOWEVER: After having produced the .bbl file,
% and prior to final submission, you *NEED* to 'insert'
% your .bbl file into your source .tex file so as to provide
% ONE 'self-contained' source file.
%
% ================= IF YOU HAVE QUESTIONS =======================
% Questions regarding the SIGS styles, SIGS policies and
% procedures, Conferences etc. should be sent to
% Adrienne Griscti (griscti@acm.org)
%
% Technical questions _only_ to
% Gerald Murray (murray@acm.org)
% ===============================================================
%
% For tracking purposes - this is V1.3 - OCTOBER 2002

% Comments at top of file (end) 

\documentclass{sig-alternate}

\begin{document}
	
% Title part (fold) 
%
% --- Author Metadata here ---
\CopyrightYear{2008}   %will cause 2005 to appear in the copyright line.
\crdata{Copyright remains with the author(s).}
\conferenceinfo{NIME08,}{ Genova, Italy}
%\CopyrightYear{2001} % Allows default copyright year (2000) to be over-ridden - IF NEED BE.
%\crdata{0-12345-67-8/90/01}  % Allows default copyright data (0-89791-88-6/97/05) to be over-ridden - IF NEED BE.
% --- End of Author Metadata ---

\title{Developing a Structured OSC Namespace for Jamoma}
\subtitle{}
%
% You need the command \numberofauthors to handle the "boxing"
% and alignment of the authors under the title, and to add
% a section for authors number 4 through n.
%
% Up to the first three authors are aligned under the title;
% use the \alignauthor commands below to handle those names
% and affiliations. Add names, affiliations, addresses for
% additional authors as the argument to \additionalauthors;
% these will be set for you without further effort on your
% part as the last section in the body of your article BEFORE
% References or any Appendices.

\numberofauthors{1}
%
% You can go ahead and credit authors number 4+ here;
% their names will appear in a section called
% "Additional Authors" just before the Appendices
% (if there are any) or Bibliography (if there
% aren't)

% Put no more than the first THREE authors in the \author command
\author{
%
% The command \alignauthor (no curly braces needed) should
% precede each author name, affiliation/snail-mail address and
% e-mail address. Additionally, tag each line of
% affiliation/address with \affaddr, and tag the
%% e-mail address with \email.
\alignauthor Authors... \\
       \affaddr{institutions...}\\
}

%%%%%%%%%%%%%%%%
\maketitle


% Title part (end) 


% Hyphenations (fold)

\hyphenation{ja-mo-ma}
% Hyphenations (end)

\begin{abstract}

The paper suggests an approach to create structured OSC messages, separating the addressing of computers and modules, from the parameters and attributes of the modules. This includes a system for querying values and parameters for creating flexible communication in modular systems. A prototype implementation is presented and discussed.

\end{abstract}

\keywords{Jamoma, OSC, standardization}



%%%%%%%%%%%%%%%%%%%%%%
%%%%%%%%%%%%%%%%%%%%%%
%%%%%%%%%%%%%%%%%%%%%%

\section{Introduction} % (fold)
\label{sec:introduction}

Jamoma\footnote{http://www.jamoma.org} is a system for developing high-level modules in the Max/MSP/Jitter environment, consisting of a recommendation and an implementation of that recommendation  \cite{Place:2006}. Most of the recent development of Jamoma has focused on improving core functionality, including flexible mapping between modules and adding ramping, function and unit conversion possibilities. 

Communication in and between Jamoma modules is being handled through the Open Sound Control (OSC)\footnote{http://www.opensoundcontrol.org} protocol. As the messaging between has grown more complex we have found that the current messaging structure of OSC is not ideal for our usage. The paper will start with an overview of some related research into development of the OSC protocol. This is followed by a suggestion for a structured approach to extending the current suggestions for OSC namespace creation. Finally, a prototype implementation in Jamoma is presented and discussed.


% section introduction (end)



%%%%%%%%%%%%%%%%%%%%%%
%%%%%%%%%%%%%%%%%%%%%%
%%%%%%%%%%%%%%%%%%%%%%

\section{Separating address and properties of a value} % (fold)
\label{sec:address_and_properites_of_a_value}



We suggest a model where it is possible to separate the recipient of addresses from the parameters to be used by the recipient (i.e. a module in Jamoma). 

This could lead to a general address space like: 

\begin{small}
\begin{verbatim}
/computer/module.N/parameter:/
\end{verbatim}
\end{small}




\subsection{The problem of OSC 1}

\subsection{Litterature Review} % (fold)
\label{sub:litterature_review}







%%%%%%%%%%%%%%%%%%%%%%
\subsubsection{Open Sound Control}

Open Sound Control (OSC) was developed in the late 1990s at CNMAT and has emerged as the de facto standard for communication among computers, sound synthesizers, and other multimedia devices in the computer music research community \cite{Wright:2003}. 


The original idea of OSC is that it is tree-structured into a hierarchy called the \emph{address space} where each of the nodes has a symbolic name and is a potential destination of OSC messages \cite{Wright:2003}. Oppositely to the well-defined structure of MIDI, the \emph{open} nature of OSC means that the address space is defined and created by the "implementor’s idea of how these features should be organized." 



“/voices/3/freq” 

This open approach has made OSC useful in a number of different situations, but it is also probably the major reason that OSC has not gained a more widespread usage outside the computer music research community. 

%%%%%%%%%%%%%%%%%%%%%%
\subsubsection{Occam}


Occam\footnote{http://www.mat.ucsb.edu/$\sim$c.ramakr/illposed/occam.html} takes OSC messages and converts them to MIDI. It exports a MIDI source to CoreMIDI which can be used in any Mac OS X application that accepts MIDI. It broadcats the existence of this OSC-to-MIDI service using Rendezvous (Zero-Conf).




%%%%%%%%%%%%%%%%%%%%%%
\subsubsection{liboscqs}

LIBOSCQS\footnote{http://liboscqs.sourceforge.net/} is a library to provide a Query System and Service Discovery for applications using the Open Sound Control (OSC) protocol\cite{Schmeder:2004oscqs}. This page contains an introduction to LIBOSCQS.


%%%%%%%%%%%%%%%%%%%%%%
\subsubsection{McGill Mapper}

\cite{Malloch:2007}


%%%%%%%%%%%%%%%%%%%%%%
\subsubsection{OSCQS}

\cite{Habets:2005}

%%%%%%%%%%%%%%%%%%%%%%
\subsubsection{Integra}

The Integra\footnote{http://www.integralive.org} project is a EU Culture 2000 project aiming at developing "a new software environment to make music with live electronics," and to modernise works that use old technology \cite{Bullock:2007}. Part of the development has been concerning creating a client/server model. 

The Integra OSC address space suggests using the : (colon) to only set messages without outputting, to avoid feedback. 

They also include a .get message to query for values. 


%%%%%%%%%%%%%%%%%%%%%%
\subsubsection{SuperCollider}

The text based computer music programming language SuperCollider is using OSC for internal communication in its client/server model. 


%%%%%%%%%%%%%%%%%%%%%%
\subsubsection{Iannix}

"Iannix \footnote{http://sourceforge.net/projects/iannix} is a graphical editor of multidimensional and multi-formal scores, a kind of poly-temporal meta-sequencer, based on the former UPIC created by Iannis Xenakis." The current implementation is built around a client/server model where all communication is handled using OSC. 

\cite{Coduys:2004}


%%%%%%%%%%%%%%%%%%%%%%
\subsubsection{TUIO}

The MTG at Pompeu Fabra has developed TUIO, an OSC address space for handling data from the Reactable \cite{Kaltenbrunner:2005}.

Copperlan

%%%%%%%%%%%%%%%%%%%%%%
\subsubsection{Open Sound Control 2.0 suggestion}

Developers of the Lemur multitouch interface presented a draft for a 2.0 version of Open Sound Control during an OSC developer meeting at NIME 2006 \cite{Jazzmutant:2006}, and has later changed the suggestion slightly \cite{Jazzmutant:2007}. 

The topic of query has been brought up and been suggested as follows: 

% subsection litterature_review (end)


%%%%%%%%%%%%%%%%%%%%%%
%%%%%%%%%%%%%%%%%%%%%%
%%%%%%%%%%%%%%%%%%%%%%



\subsection{Classes} % (fold)
\label{sub:classes}

% subsection classes (end)



% section address_and_properites_of_a_value (end)



%%%%%%%%%%%%%%%%%%%%%%
%%%%%%%%%%%%%%%%%%%%%%
%%%%%%%%%%%%%%%%%%%%%%

\section{Jamoma as a prototype implementation} % (fold)
\label{sec:prototype_implementation}

The general concepts introduced in the previous section has formed the basis for the implementation of a standardized namespace addressing properties of values in the Jamoma framework for Max/MSP \cite{Place:2006}. The following description of the implementation is meant to serve as an illustration of the concept, as well as indicating how it might provide the user with extended and structured control of available values.

Jamoma distinguish between module \emph{parameters} and \emph{messages}. Parameters alter the state of the module. When querying or setting the state of the module, parameter values need to be retrieved or set. In contrast messages are stateless. One example of a message would be a request to open a reference file for the module. Apart from the difference in terms of state, messages and parameters behave the same, and the following discussion is equally valid for both.

\subsection{The colon separator} % (fold)
\label{sub:the_colon_separator}

In addition to the ASCII symbols already reserved for specific purposes within the OSC protocol \cite{Wright:1997}, we introduce the colon "\texttt{:}" as a separator between the OSC address of the parameter and the namespace for accessing the properties of the parameter:

\texttt{<parameter address> <value>}

\texttt{<parameter address>:<property address> <value>}

The former message sett the value itself, while the latter sets a property of the message. The address of the parameter will be omitted in the following examples for the sake of brevity; e.g. \texttt{/computer/module/parameter:property} will be abbreviated as \texttt{:property}.

% subsection the_colon_separator (end)

\subsection{Controlling the value itself} % (fold)
\label{sub:controlling_the_value_itself}

In addition to setting the value directly, it can be set and retrieved as a property. Value can also be stepwise increased or decresed, and the size of the steps is itself a property:

\texttt{:value}

\texttt{:value/get}

\texttt{:inc}

\texttt{:dec} 

\texttt{:value/stepsize}

\texttt{:value/stepsize/get}

% subsection controlling_the_value_itself (end)



\subsection{Parameter or message type} % (fold)
\label{sub:type}

The type of the value can be specified. Possible types are \emph{none}, \emph{boolean}, \emph{integer}, \emph{float}, \emph{symbols} and \emph{lists}. The \emph{none} type is only valid for messages. Some of the properties below will only be valid for certain types of values. The type property is accessed thus:

\texttt{:type}

\texttt{:type/get}

% subsection type (end)


\subsection{Controlling the range} % (fold)
\label{sub:range}

For integer, float and list values a range can be specified. This can be useful for setting up autoscaling mappings from one parameter value to another, or for clipping the output range. The clipping property can be \emph{none}, \emph{low}, \emph{high} or \emph{both}. The range properties are accessed thus:

\texttt{:range}

\texttt{:range/get}

\texttt{:range/clipmode}

\texttt{:range/clipmode/get}

% subsection range (end)


\subsection{Filtering of repetitions} % (fold)
\label{sub:filtering_of_repetitions}

It is soemtimes useful to filter repetitions to avoid redundant processing. The \emph{boolean} repetitions property is accessed thus:

\texttt{:repetitions}

\texttt{:repetitions/get}

% subsection filtering_of_repetitions (end)


\subsection{Ramping to new values} % (fold)
\label{sub:ramping_to_new_values}

The ability to smoothly move from one value to another is fundamental to any kind of transition ands transformation of musical or artistic material. Jamoma offers the possibilty of ramping from the current to a new value in a set amount of time. While the OSC message

\texttt{/myComputer/myModule/myParameter 1.0}

will set the parameter value to $1.0$ immediately, the message

\texttt{/myComputer/myModule/myParameter 1.0 ramp 2000}

will cause the parameter to ramp to $1.0$ over 2000 milliseconds. Ramping in Jamoma works with messages and parameters of type integer, float and list.

Jamoma offers vastly extended possibilities in how ramping can be done as compared to Max. In Jamoma the process of ramping is split into two components: A driving mechanism forcing calculations of new values during the ramp, and a set of functions describing the curve of the ramping. Both components are implemented as C++ APIs, and can easily be extended with new functionalities.

\subsubsection{The Jamoma RampLib API} % (fold)
\label{ssub:the_ramp_lib}

The Jamoma RampLib API provides a means by which to create and use \emph{Ramp Units} in Jamoma.  A Ramp Unit is a self-contained algorithm that can slide from an existing value to a new value over a specified amount of time according to different timing mechanisms. Each Rasmp Unit is implemented in the form of two C++ files: a source file and a header file that provides an interface for the source file. Currently four such Ramp Units are implemented:

\begin{itemize}

	\item \emph{none} - jumps immediately to the new value. Typically used for values where ramping do not make sense.

	\item \emph{scheduler} - use the Max internal clock to generate new values at fixed time intervals.

	\item \emph{queue} - ramping using the Max queue, updating values whenever the processor has free capacity to do so.

	\item \emph{async} - only calculate new values when requested to do so. This might be used in video processing modules to calculate fresh values immediately before processing the next video image or matrix.
	
\end{itemize}

When a new ramp is started, the Ramp Unit internally use a normalized ramping value, increasing linearly from $0.0$ to $1.0$ over the duration of the ramp. Whenever the Ramp Unit is to provide a new value, it updates the normalized ramping value, and pass it to a Function Unit as described in Section~\ref{ssub:the_function_lib}. The normalized value returned is then scaled to the range defined by the start and end values for the ramp, and passed on to the module.


% subsubsection the_ramp_lib (end)


\subsubsection{The Jamoma FunctionLib API} % (fold)
\label{ssub:the_function_lib}

The Jamoma FunctionLib API provides normalized mappings of values $x \in [0,1]$ to $y \in [0,1]$ according to functions $y = f(x)$. The FunctionLib can easily be expanded by introducing new functions in the form of two C++ files: a source file and a header file that provides an interface for the source file.

Currently five functions are implemented: 

\begin{itemize}

	\item Linear: $y = x$.

	\item Cosine: $y = - \frac{1}{2} * cos(x * \pi ) + \frac{1}{2} $.

	\item Lowpass series: $y[n] = y[n-1] * k + x[n] * (1-k)$, \\ where $k$ is a feedback coefficient.

	\item Power function: $ y = x^{k} $, where the parameter $k$ can be set.

	\item Hyperbolic tangens: $ y = c * (tanh(a*(x-b)) - d) $, \\ where coefficients $a$, $b$, $c$, $d$ depends on the width and offset of the curve.
	
\end{itemize}

There are plans to introduce exponential functions.


% subsubsection the_function_lib (end)



\subsubsection{OSC namespace for ramping properties} % (fold)
\label{ssub:osc_namespace_for_ramping_properties}

\texttt{:ramp/drive}

\texttt{:ramp/drive/get}

\texttt{:ramp/drive/dump}



\texttt{:ramp/drive/parameter}

\texttt{:ramp/drive/parameter/get}

\texttt{:ramp/drive/parameter/dump}



\texttt{:ramp/function}

\texttt{:ramp/function/get}

\texttt{:ramp/function/dump}



\texttt{:ramp/function/parameter}

\texttt{:ramp/function/parameter/get}

\texttt{:ramp/function/parameter/dump}



\texttt{:ramp/function/parameter/names/dump}

\texttt{:ramp/function/parameter width}

\texttt{:ramp/function/parameter/get width}




% subsubsection osc_namespace_for_ramping_properties (end)


% subsection ramping_to_new_values (end)

\subsection{Description} % (fold)
\label{sub:description}

The description property is a string providing a text description of the parameter. In Jamoma this is used for auto-generating online documentation of the modules. It can also be used for building modules that retrieve the total namespace of all Jamoma modules used, and provide interactive documentation of available parameters. The descriptions property is accessed as:

\texttt{:description}

\texttt{:description/get}

% subsection description (end)



%%%%%%%%%%%%%%%%%%%%%%

\subsection{Separators} % (fold)
\label{sub:separators}


Having discussed different ways of designing a message like this one, we seem to have agreed that the meaning of the different separating signs used need to be precisely defined:

\begin{itemize}

	\item. (dot): Used to indicate instances of a specific class.
	\item / (slash): Used to indicate branching according to the OSC specification.
	\item : (colon): Colon is used to split the total OSC message into two parts. The first part is describing where you want to go. The second part describe what you want to access there.
\end{itemize}

% subsection separators (end)



%%%%%%%%%%%%%%%%%%%%%%
\subsection{Namespace structure}
\texttt{:value/get}

%% This should probably be in a table if included, but just adding it here as a draft

		Message                                               Category       Description

		- [ ] /module/parameter
		    - [ ] General for the module
		        - [ ] :namespace/dump
		        - [ ] :dump/values
		    - [ ] value stuff                                                These ones are not fixed.
		        - [X] :value/get                              value          get the value of the parameter
		        - [X] :value                                  value          set the value (why would you want to do it this way?)
		        - [x] :value/range                            range
		        - [x] :value/range/get                        range
		        - [x] :value/range/clipmode                   range
		        - [x] :value/range/clipmode/get               range
		        - [X] :value/step/inc                         value          Increase by one step
		        - [X] :value/step/dec                         value          Decrease by one step
		        - [X] :value/step/size                        value          Set the step size
		        - [X] :value/step/size/get                    value          Get the step size
		        - [ ] :value/unit
		        - [x] :value/type                             type
		        - [x] :value/type/get                         type
		    - [ ] Conditioning
		        - [x] :repetitions                            repetitions
		        - [x] :repetitions/get                        repetitions
		    - [ ] ramp stuff - including drive and function
		
		        - [x] :ramp/drive                             ramp/drive     Select what mechanism is driving the ramp
		        - [x] :ramp/drive/get                         ramp/drive
		        - [x] :ramp/drive/dump                        ramp/drive	 Dump all possible driving mechanisms available
		
		        - [x] :ramp/drive/parameter                   ramp/drive     Pass additional parameters to the driving mechanism
		        - [x] :ramp/drive/parameter/get               ramp/drive
		        - [x] :ramp/drive/parameter/dump              ramp/drive	
		
		        - [x] :ramp/function                          ramp/drive     Select what mapping function to use for the ramp
		        - [x] :ramp/function/get                      ramp/function
		        - [x] :ramp/function/dump                     ramp/function	 
		
		        - [x] :ramp/function/parameter                ramp/function  Pass additional parameters to the function
		        - [x] :ramp/function/parameter/get            ramp/function
		        - [x] :ramp/function/parameter/dump           ramp/function	

		        - [x] :ramp/function/parameter/names/dump     ramp/function  Get the name of all parameters for the currently used function
		        - [x] :ramp/function/parameter width          ramp/function  This is an example (valid for the tanh function)
		        - [x] :ramp/function/parameter/get width      ramp/function  This is an example (valid for the tanh function)
		    - [ ] ui
		        - [ ] :ui/freeze                              ui
		        - [ ] :ui/freeze/get                          ui
				- [ ] :ui/refresh	                          ui



% section prototype_implementation (end)



%%%%%%%%%%%%%%%%%%%%%%
%%%%%%%%%%%%%%%%%%%%%%
%%%%%%%%%%%%%%%%%%%%%%

\section{Discussion and further work} % (fold)
\label{sec:discussion_and_further_work}

% section discussion_and_further_work (end)

4 Ramp Units times 5 Function Units = 20 ramping modes

Ramp Units can be used for other scheduled processes as well

Possibility of expanding Ramp Units as low frequency oscillators

Function Units can be used elsewhere, e.g. for mapping

Audio rate ramp unit.

DataspaceLib


%\end{document}  % This is where a 'short' article might terminate



%%%%%%%%%%%%%%%%%%%%%%
%%%%%%%%%%%%%%%%%%%%%%
%%%%%%%%%%%%%%%%%%%%%%

%ACKNOWLEDGMENTS are optional
\section{Acknowledgments} % (fold)
\label{sec:acknowledgments}

All Jamoma developers and users for valuable contributions. 

% section acknowledgments (end)



%%%%%%%%%%%%%%%%%%%%%%
%%%%%%%%%%%%%%%%%%%%%%
%%%%%%%%%%%%%%%%%%%%%%

% Bibliography (fold)
%
% The following two commands are all you need in the
% initial runs of your .tex file to
% produce the bibliography for the citations in your paper.
\bibliographystyle{abbrv}
\bibliography{jamoma-nime2008}  % the name of the Bibliography in this case
% You must have a proper ".bib" file
%  and remember to run:
% latex bibtex latex latex
% to resolve all references
%
% Bibliography (end)


\balancecolumns % GM July 2000
% That's all folks!
\end{document}
